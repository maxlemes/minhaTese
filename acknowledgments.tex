
\newpage
\chapter*{Acknowledgments}
 Gostaria de agradecer em primeiro lugar a Deus, que sempre proporcionou-me uma vida cheia de ben\c c\~aos e conquistas. Ao meu orientador Prof. Dr. Orizon Pereira Ferreira, por todos os ensinamentos e direcionamentos, pela infind\'avel paci\^encia e admir\'avel compreens\~ao, al\'em do seu determinante aux\'ilio para a concretiza\c c\~ao desta tese. Ao meu coorientador Prof. Dr. Glaysdston de Carvalho Bento, pelos ensinamentos e pela enorme contribui\c c\~ao para a realiza\c{c}\~ao deste trabalho. Aos professores Jo\~ao Xavier da Cruz Neto, Maicon Marques Alves e Lu\'is Rom\'an Lucambio P\'erez, por aceitarem participar da minha banca de defesa e pelas valiosas sugest\~oes. Estendo meus agradecimentos \`a CAPES pela grande ajuda oferecida com a bolsa de estudos de doutorado, \`a Universidade Federal de Goi\'as (UFG), pela aceita\c{c}\~ao no seu programa de doutorado como aluno regular e os conhecimentos adquiridos. Aos professores da UFG pelo apoio e dedica\c{c}\~ao, em especial aos que integram o grupo de Otimiza\c{c}\~ao, e a todos os funcion\'arios da UFG. Finalmente, devo muita gratid\~ao a minha esposa (Luama), irm\~aos (Adriano e L\'ivia), m\~ae (Gleice), av\'o (Dalva) e a todos os amigos que me acolheram aqui em Goi\^ania e aos queridos amigos do Piau\'i; voc\^es foram muito importantes para mim todos esses anos.

 \medskip

 \noindent
\newpage