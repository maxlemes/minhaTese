\begin{tikzpicture}[scale = 1, rotate = -0]

	% adicionando as linhas de grade
	%\draw[help lines] (-1,-1) grid (15,11);

	% definindo os pontos
	\coordinate (A) at (1,1.5);
	\coordinate (C) at (2,8);
	\coordinate (D) at (3,0);

	\coordinate (V) at (9,6);
	\coordinate (P) at (6,4);
	\coordinate (W) at ($(V)!1.2!-37:(P)$);
	\coordinate (U) at (3,5);
	\coordinate (M) at ($(P)!3cm!90:(V)$);

	% definindo as regioes 

	% curva Bezier onde (270:-1) define o angulo e o sentido que passa pelo ponto A
	\def\regionA{
		(1,1.5)
		.. controls ++(270:-1) and ++( 30:-1) .. (2,4)
		.. controls ++( 30: 1) and ++(120: 1) .. (6,4)
		.. controls ++(120:-1) and ++(180:-1) .. (3,0)
		.. controls ++(180: 1) and ++(270: 1) ..
		cycle;
	}

	% definindo um círculo com centro V e raio 5 cm
	\def\regionB{(V) circle (4.5cm);}

	% desenhando o segmento uP e Pv
	\draw [dashed,line width=0.8pt] (U) -- (V);
	\draw [dashed,line width=0.8pt] (V) -- (W);
	\draw[dashed,gray,line width=0.5pt]\regionB

	% colorindo a interseçao
	\begin{scope}[shift={(-1.5cm,5.5cm)},rotate=-30]
		\clip  % definindo a região fora da \regionA e dentro da \regionB
		\regionA
		\fill[fill=cyan, opacity=0.5]
		\regionB

		% definindo a fronteira da \regionB no interior da \regionA
		\draw[line width=0.8pt]\regionB
	\end{scope}

	% desenhando a fronteira da \regionA
	\draw[line width=0.8pt,shift={(-1.5cm,5.5cm)},rotate=-30]\regionA


	%\draw[line width=0.8pt](V) circle (4.47cm);


	% adicionando os pontos ao gráfico
	\node[above] at (C) {$C$};

	\node[below] at (W) {$w$};
	\filldraw (W) circle(1pt);
	\node[above] at (V) {$v$};
	\filldraw (V) circle(1pt);
	\node[left] at (U) {$u$};
	\filldraw (U) circle(1pt);

	% adicionando as legendas

	\matrix [draw,anchor=north west] at (current bounding box.north west) {
		\node {$\bullet \;\; w \in {\cal P}_{C,\zeta}^{D}(u, v)$}; \\
	};

\end{tikzpicture}