\begin{tikzpicture}[scale = 1]
	\makeatletter
	\def\convertto#1#2{\strip@pt\dimexpr #2*65536/\number\dimexpr 1#1}
	\makeatother

	% adicionando as linhas de grade
	%\draw[help lines] (-1,-1) grid (12,9);

	% definindo os pontos
	\coordinate (A) at (1,1.5);
	\coordinate (C) at (2,4);
	\coordinate (D) at (3,0);

	\coordinate (V) at (8.4,5.6);
	\coordinate (P) at (5,4.5);
	\coordinate (W) at ($(V)!1.05!2:(P)$);

	\coordinate (W1) at ($(V)!1.06!-11:(P)$);
	\coordinate (W2) at ($(V)!1.1!-2:(P)$);
	\coordinate (W3) at ($(V)!1.09!7:(P)$);
	\coordinate (W4) at ($(V)!1.03!16:(P)$);
	\coordinate (W5) at ($(V)!0.97!21:(P)$);
	\coordinate (U) at (5,3);
	\coordinate (M) at ($(P)!3cm!90:(V)$);


	%--Computing the distance between (U) and (W)
	%Creating a math coordinate
	\tikzmath{coordinate \C;
		%Storing coordinates difference
		\C = (W)-(U);
		%Computing the length of C = (Cx,Cy) from its components Cx and Cy
		%Note the length \distAB is in points (pt)
		\distWU = sqrt((\Cx)^2+(\Cy)^2);
		% coordinate \D;
		% %Storing coordinates difference
		% \D = (W)-(V);
		% %Computing the length of C = (Cx,Cy) from its components Cx and Cy
		% %Note the length \distAB is in points (pt)
		% \distWV = sqrt((\Dx)^2+(\Dy)^2);
		%Convert back to centimeters
		\distAB = \convertto{cm}{\distWU pt};
		\dist =  0.5*\distAB;
	}

	% definindo o ponto N acrescente draw dentro do colchete para desenhar o circulo
	\node(Circle) at (W) [circle through=($(W)!0.5!(U)$)] {};
	\coordinate (N) at (intersection 1 of W--V and Circle);


	% definindo as regioes 

	% curva Bezier onde (270:-1) define o angulo e o sentido que passa pelo ponto A
	\def\regionA{
		(1,1.5)
		.. controls ++(270:-1) and ++( 30:-1) .. (2,4)
		.. controls ++( 30: 1) and ++(120: 1) .. (6,4)
		.. controls ++(120:-1) and ++(180:-1) .. (3,0)
		.. controls ++(180: 1) and ++(270: 1) ..
		cycle;
	}

	% definindo um círculo com centro V e raio 5 cm

	\def\regionB{
		(W1)
		.. controls ++(180:0.1) and ++(210:-0.2) .. (W2)
		.. controls ++(210:0.2) and ++(240:-0.2) .. (W3)
	%	.. controls ++(250:0.5) and ++(300:-0.2) .. (W4)
		.. controls ++(240:0.2) and ++(320:-0.2) .. (W5)
		.. controls ++(320:0.2) and ++(180:-0.1) ..
		cycle;

	}
	%\draw[line width=0.8pt]\regionB

	% colorindo a interseçao
	\begin{scope}[shift={(5cm,-1cm)},rotate=70]
		\clip  % definindo a região fora da \regionA e dentro da \regionB
		\regionA
		\fill[fill=cyan, opacity=0.5]
		\regionB

		% definindo a fronteira da \regionB no interior da \regionA
		\draw[line width=0.8pt]\regionB
		%\draw[fill=red, opacity = 0.5,line width=0.8pt,shift={(5cm,-1cm)},rotate=70] (V) circle [radius=3.51cm];
		%\draw[fill=red, opacity = 0.5,line width=0.8pt,shift={(5cm,-1cm)},rotate=70] (V) circle [radius=4.28cm];
	\end{scope}

	% desenhando a fronteira da \regionA
	\draw[line width=0.8pt,shift={(5cm,-1cm)},rotate=70]\regionA

	% \draw[dashed,gray,line width=0.5pt,shift={(5cm,-1cm)},rotate=70] (V) circle (1.15cm);

	% desenhando a reta perpendicular
	\draw [line width=0.8pt] (N) -- ($(N)!4cm! 90:(V)$) node[right] {$\pi$};
	\draw [line width=0.8pt] (N) -- ($(N)!4cm!-90:(V)$);

	%\draw[line width=0.8pt](V) circle (4.47cm);

	% desenhando o segmento uP e Pv
	\draw [dashed, line width=0.8pt] (U) -- (W);
	\draw [dashed, line width=0.8pt] (V) -- (W);


	% % Adicionando os braços
	% \draw[decorate,red,decoration={brace,amplitude=2mm},line width=0.8pt](W)--node[left,xshift=-0.1cm,yshift=0.4cm] {$d(w,\pi)$}(N);
	% \draw[decorate,blue,decoration={brace,amplitude=2mm},line width=0.8pt](U)--node[anchor=south east] {$d(u,w)$}(W);


	% adicionando os pontos ao gráfico
	\node[anchor=south east] at (C) {$C$};

	\node[anchor=south] at (W) {$w$};
	\filldraw (W) circle(1pt);
	\node[above] at (V) {$v$};
	\filldraw (V) circle(1pt);
	\node[anchor=north west] at (U) {$u$};
	\filldraw (U) circle(1pt);
	% \node[anchor=north west] at (E) {$E$};
	% \filldraw (E) circle(1pt);
	% \node[anchor=south] at (W1) {$w1$};
	% \filldraw (W1) circle(1pt);
	% \node[anchor=south] at (W2) {$w2$};
	% \filldraw (W2) circle(1pt);
	% \node[anchor=south] at (W3) {$w3$};
	% \filldraw (W3) circle(1pt);
	% \node[anchor=south] at (W4) {$w4$};
	% \filldraw (W4) circle(1pt);
	% \node[anchor=south] at (W5) {$w5$};
	% \filldraw (W5) circle(1pt);


	% adicionando as legendas
	% \matrix [draw,anchor=south east] at (current bounding box.north east) {
	% 	\node {$\bullet \;\; w \in {\cal R}_{C,\gamma}^{D}(u, v)$}; \\
	% };

\end{tikzpicture}