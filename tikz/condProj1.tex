\begin{tikzpicture}[scale = 1, rotate = 0]
	\makeatletter
	\def\convertto#1#2{\strip@pt\dimexpr #2*65536/\number\dimexpr 1#1}
	\makeatother

	% adicionando as linhas de grade
	%\draw[help lines] (-1,-1) grid (12,9);

	% definindo os pontos
	\coordinate (A) at (1,1.5);
	\coordinate (C) at (2,8);
	\coordinate (D) at (3,0);

	\coordinate (V) at (9,6);
	\coordinate (P) at (5.7,6);
	\coordinate (W) at ($(V)!1.25!-8:(P)$);
	\coordinate (U) at (3,6);
	\coordinate (Y) at (3,4.5);

	\coordinate (M) at ($(P)!3cm!90:(V)$);


	%--Computing the distance between (U) and (W)
	%Creating a math coordinate
	\tikzmath{coordinate \C;
		%Storing coordinates difference
		\C = (W)-(U);
		%Computing the length of C = (Cx,Cy) from its components Cx and Cy
		%Note the length \distAB is in points (pt)
		\distWU = sqrt((\Cx)^2+(\Cy)^2);
		% coordinate \D;
		% %Storing coordinates difference
		% \D = (W)-(V);
		% %Computing the length of C = (Cx,Cy) from its components Cx and Cy
		% %Note the length \distAB is in points (pt)
		% \distWV = sqrt((\Dx)^2+(\Dy)^2);
		%Convert back to centimeters
		\distAB = \convertto{cm}{\distWU pt};
		\dist =  0.5*\distAB;
	}

	% definindo o ponto Wt
	\node(Circle) at (W) [circle through=($(W)!0.5!(U)$)] {};
	\coordinate (Wt) at (intersection 1 of W--V and Circle);


	% definindo as regioes 

	% curva Bezier onde (270:-1) define o angulo e o sentido que passa pelo ponto A
	\def\regionA{
		(1,1.5)
		.. controls ++(270:-1) and ++( 30:-1) .. (2,4)
		.. controls ++( 30: 1) and ++(120: 1) .. (6,4)
		.. controls ++(120:-1) and ++(180:-1) .. (3,0)
		.. controls ++(180: 1) and ++(270: 1) ..
		cycle;
	}

	% definindo um círculo com centro V e raio 5 cm
	\def\regionB{(V) circle (2.5cm);}

	% colorindo a interseçao
	% \begin{scope}
	% 	\clip  % definindo a região fora da \regionA e dentro da \regionB
	% 	\regionA
	% 	\fill[fill=purple]
	% 	\regionB

	% 	% definindo a fronteira da \regionB no interior da \regionA
	% 	\draw[line width=0.8pt]\regionB
	% \end{scope}

	% desenhando a fronteira da \regionA
	\draw[line width=0.8pt,shift={(-1.5cm,5.5cm)},rotate=-30]\regionA

	% desenhando a reta perpendicular
	\draw [line width=0.8pt] (Wt) -- ($(Wt)!3cm! 90:(V)$) node[right] {$\pi$};
	\draw [line width=0.8pt] (Wt) -- ($(Wt)!3cm!-90:(V)$);

	%\draw[line width=0.8pt](V) circle (4.47cm);

	% Mark the angle uPv
	\begin{scope}[shift={(-1.5cm,5.5cm)},rotate=-30]
		\clip (Y) -- (Wt) -- (V);
		\fill[red, opacity=0.5, draw=black] (Wt) circle (5mm);
		\draw (Wt) circle (5mm);
	\end{scope}

	% desenhando o segmento uP e Pv
	\draw [dashed, line width=0.8pt] (U) -- (W);
	\draw [dashed, line width=0.8pt] (V) -- (W);
	\draw [dashed, line width=0.8pt] (Y) -- (Wt);

	% Adicionando a norma e a seta indicativa
	\draw[red,-latex, smooth,line width=0.8pt] (5,8) .. controls (5.5,7.5) .. (5.5,6.8);
	\node[red,anchor=south] at (5,8) {$\|w_t-w\|_D$};

	% Adicionando os braços
	\draw[decorate,red,decoration={brace,amplitude=2mm},line width=0.8pt](W)--(Wt);
	\draw[decorate,blue,decoration={brace,amplitude=2mm},line width=0.8pt](U)--node[anchor=south,yshift=0.3cm] {$\|w-u\|_D$}(W);


	% adicionando os pontos ao gráfico
	\node[above] at (C) {$C$};

	\node[anchor=north west] at (W) {$w$};
	\filldraw (W) circle(1pt);
	\node[above] at (V) {$v$};
	\filldraw (V) circle(1pt);
	\node[anchor=north west] at (U) {$u$};
	\filldraw (U) circle(1pt);
	\node[anchor=south west] at (Wt) {$w_t$};
	\filldraw (Wt) circle(1pt);
	\node[anchor=south] at (Y) {$y$};
	\filldraw (Y) circle(1pt);

	% adicionando as legendas

	\matrix [draw, anchor=west] at (current bounding box.north west) {
		\node {$\bullet \;\; \sqrt{\gamma} \|w-u\|_D \leq \|w_t-w\|_D$}; \\
		\node {$\bullet \;\; w \in {\cal R}_{C,\gamma}^{D}(u, v)$}; \\
	};

\end{tikzpicture}