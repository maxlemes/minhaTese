\begin{tikzpicture}[scale = 1, rotate = 0, blacknode/.style={shape=circle, fill=black, line width=0},]

	% adicionando as linhas de grade
	%\draw[help lines] (0,0) grid (15,10);

	% definindo os pontos
	\coordinate (A) at (1,1.5);
	\coordinate (C) at (2,8);
	\coordinate (D) at (3,0);

	\coordinate (V) at (9,6);
	\coordinate (P) at (5.7,6);
	\coordinate (U) at (3,5);
	\coordinate (M) at ($(P)!3cm!90:(V)$);

	% definindo as regioes 

	% curva Bezier onde (270:-1) define o angulo e o sentido que passa pelo ponto A
	\def\regionA{
		(1,1.5)
		.. controls ++(270:-1) and ++( 30:-1) .. (2,4)
		.. controls ++( 30: 1) and ++(120: 1) .. (6,4)
		.. controls ++(120:-1) and ++(180:-1) .. (3,0)
		.. controls ++(180: 1) and ++(270: 1) ..
		cycle;
	}

	% definindo um círculo com centro V e raio 5 cm
	\def\regionB{(V) circle (5cm);}

	% % colorindo a interseçao
	% \begin{scope}
	% 	\clip  % definindo a região fora da \regionA e dentro da \regionB
	% 	\regionA
	% 	\fill[fill=purple]
	% 	\regionB

	% 	% definindo a fronteira da \regionB no interior da \regionA
	% 	\draw[line width=0.8pt]\regionB
	% \end{scope}

	% desenhando a fronteira da \regionA
	\draw[line width=0.8pt,shift={(-1.5cm,5.5cm)},rotate=-30]\regionA


	% mark right angle
	\tkzMarkRightAngle[fill=red, opacity=0.5](V,P,M);

	% desenhando a reta perpendicular
	\draw [line width=0.8pt] (P) -- ($(P)!3cm! 90:(V)$) node[right] {$\pi$};;
	\draw [line width=0.8pt] (P) -- ($(P)!3cm!-90:(V)$);

	% Mark the angle uPv
	\begin{scope}[shift={(-1.5cm,5.5cm)},rotate=-30]
		\clip (U) -- (P) -- (V);
		\fill[cyan, opacity=0.5, draw=black] (P) circle (5mm);
		\draw (P) circle (5mm);
	\end{scope}

	% adicionando o angulo theta
	\node at ($(P)+(-30:7mm)$) {$\theta$};

	%\draw[line width=0.8pt](V) circle (4.47cm);

	% desenhando o segmento uP e Pv
	\draw [dashed, line width=0.8pt] (U) -- (P);
	\draw [dashed, line width=0.8pt] (V) -- (P);


	% adicionando os pontos ao gráfico
	\node[above] at (C) {$C$};

	\node[anchor=south east] at (P) {$w$};
	\filldraw (P) circle(1pt);
	\node[anchor=south] at (V) {$v$};
	\filldraw (V) circle(1pt);
	\node[anchor=south] at (U) {$y$};
	\filldraw (U) circle(1pt);

	% adicionando as legendas
	%\node at (4,-1)  {$w = {\cal P}_{C}^{D}(v)$};

	\matrix [draw,anchor=south east] at (current bounding box.north east) {
	\node {$\bullet \;\; w = {\cal P}_{C}^{D}(v)$};\\
	};

\end{tikzpicture}