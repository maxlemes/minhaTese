\chapter{Preliminaries}  \label{chap:Prel}
%%%%%%%%%%%%%%%%%%%%%%%%%%%%%%%%%%%%%%%%%%%%%%%%%%%%%%%%%%

\section{Basic results}

In this chapter, we introduce  some notation and results used throughout our presentation.  First we  consider the  index set  ${\mathbb{N}}:=\{0,1,2,\ldots\}$,  the usual inner  product  $\langle \cdot,\cdot \rangle$ in $\mathbb{R}^n$, and the associated Euclidean norm    $\|\cdot\|$.
Let  $f:\mathbb{R}^n \to \mathbb{R}$ be a differentiable function and $C \subseteq \mathbb{R}^n$. The  gradient $\nabla f$ of $f$ is said to be {\it Lipschitz continuous} in $C$ with constant $L>0$ if $\|\nabla f(x)-\nabla f(y)\|\leq L \|x-y\|$, for all~$x, y\in C$. Combining this definition with the fundamental theorem of calculus, we obtain the following result whose proof can be found in \cite[Proposition A.24]{Bertsekas1999}.

\begin{lemma} \label{Le:derivlipsch}
	Let $f:\mathbb{R}^n \to \mathbb{R}$ be a differentiable function and $C \subseteq \mathbb{R}^n$. Assume that $\nabla f$  is Lipschitz continuous in C with constant $L>0$. Then,
	\[
		f(y) - f(x) - \langle \nabla f(x), y-x \rangle \leq (L/2)\|x-y\|^2,
	\]
	for all~ $x, y\in C$.
\end{lemma}
%Let $f:\mathbb{R}^n \to \mathbb{R}$ be a differentiable function and $C \subseteq \mathbb{R}^n$ be a convex set. The function $f$ is convex on $C$, if $f(y) \geq  f(x) + \langle \nabla f(x), y-x \rangle$, for all $x, y\in C$.
Assume that $C$ is a convex set. The function $f$ is said to be \textit{convex} on $C$, if
\[
	f(y) \geq  f(x) + \langle \nabla f(x), y-x \rangle,
\]
for all $x, y\in C$. We recall  that a point ${\bar{x}} \in C$ is a {\it stationary point} for problem \eqref{eq:OptP} if
\begin{equation} \label{eq:StatPoint}
	\langle \nabla f({\bar{x}}),  x-{\bar{x}}\rangle \geq 0, \qquad \forall ~ x\in  C.
\end{equation}
Consequently, if $f$ is a convex function  on $C$, then  \eqref{eq:StatPoint} implies that  $\bar{x} \in \Omega^*$.  We will now present some useful concepts  for the analysis of the sequence generated by the scaled  gradient method, for more details, see \cite{CombettesVu2013}.    For that,  let  $D$ be a $n\times n$ positive definite matrix and $\| \cdot \|_{D} : \mathbb{R}^{n}\rightarrow \mathbb{R}$ be  the norm  defined by
\begin{equation} \label{def:normaD}
	\|d\|_{D}:=\sqrt{\left\langle D d,d\right\rangle},\quad \forall d\in \mathbb{R}^{n}.
\end{equation}
For a fixed  constant $\mu \geq 1$,  {\it denote by  ${\cal D}_{\mu}$  the set of symmetric positive definite matrices $n\times n$ with all eigenvalues contained in the interval $[\frac{1}{\mu}, \mu]$}.  The set ${\cal D}_{\mu}$   is compact. Moreover,  for each $D\in {\cal D}_{\mu}$, it follows that $D^{-1}$ also belongs to $ {\cal D}_{\mu}$. Furthermore,  due to $D\in {\cal D}_{\mu}$,  by \eqref{def:normaD}, we obtain
\begin{equation} \label{eq:pnv}
	\frac{1}{\mu}\|d\|^2\leq \|d\|^2_{D}\leq \mu \|d\|^2, \qquad \forall d\in \mathbb{R}^n.
\end{equation}

Let us recall the the concept of  sequence  quasi-Fej\'er  monotone to a set, introduced in  \cite{CombettesVu2013}.
\begin{definition} \label{def:QuasiFejer}
	Let $(y^k)_{k\in\mathbb{N}}$ be a sequence in $\mathbb{R}^n$ and   $(D_k)_{k\in\mathbb{N}}$ be  a sequence in ${\cal D}_{\mu}$.  The sequence $(y^k)_{k\in\mathbb{N}}$ is said to be quasi-Fej\'er  monotone to a set $W\subset \mathbb{R}^n$ with respect to  $(D_k)_{k\in\mathbb{N}}$ if, there exists  a sequence $(\eta_k)_{k\in\mathbb{N}}\subset [0, +\infty)$ such that $\sum_{k\in \mathbb{N}}\eta_k<\infty$ and for  all $w\in W$, there exists a sequence $(\epsilon_k)_{k\in\mathbb{N}}\subset[0, +\infty)$ such that  $\sum_{k\in \mathbb{N}}\epsilon_k<\infty$, and
	\[
		\|y_{k+1}-w\|_{D_{k+1}}^2\leq (1+\eta_k) \|y^k-w\|_{D_k}^2+\epsilon_k,
	\]
	for    all $k\in \mathbb{N}$.
\end{definition}

The following lemma is useful to study the  quasi-Fej\'er  monotone sequence, its prove can be found in \cite[Lemma 2.2.2]{PolyakLivro1987}.
\begin{lemma} \label{le:pl}
	Let $(\alpha_k)_{k\in\mathbb{N}}$, $(\eta_k)_{k\in\mathbb{N}}$ and $(\epsilon_k)_{k\in\mathbb{N}}$ be a sequences in $[0, +\infty)$  such that $\sum_{k\in \mathbb{N}}\eta_k<\infty$ and $\sum_{k\in \mathbb{N}}\epsilon_k<\infty$. Assume that $\alpha_{k+1}\leq (1 +\eta_k)\alpha_k+\epsilon_k$, for all $k\in {\mathbb N}$. Then, $(\alpha_k)_{k\in\mathbb{N}}$ converges.
\end{lemma}

The main property of  quasi-Fej\'er  monotone sequences is stated in the following. Its proof can be found in \cite[Proposition~3.2 and Theorem 3.3]{CombettesVu2013}. For sake of completeness, we include it here.

\begin{theorem}\label{teo.qf}
	Let $(y^k)_{k\in\mathbb{N}}$ be a sequence in $\mathbb{R}^n$ and   $(D_k)_{k\in\mathbb{N}}$ be  a sequence in ${\cal D}_{\mu}$ such that $\lim_{k\rightarrow\infty}D_k={\bar D}$.   If $(y^k)_{k\in\mathbb{N}}$ is quasi-Fej\'er  monotone to a nonempty set $W\subset  \mathbb{R}^n$ with respect to $(D_k)_{k\in\mathbb{N}}$ then,  for each  $w\in W$,    the sequence  $(\|y^{k}-w\|_{D_{k}})_{k\in\mathbb{N}}$ converges. Furthermore, $(y^k)_{k\in\mathbb{N}}$ is bounded and, if each  cluster point of $(y^k)_{k\in\mathbb{N}}$ belongs to $W$, then there exists ${\bar y}\in W$ such that $\lim_{k\rightarrow\infty}y^k={\bar y}$.
\end{theorem}

\begin{proof}
	Take $w\in W$ and define the sequence $(\alpha_k)_{k\in\mathbb{N}}$, where $\alpha_k:=\|y^{k}-w\|_{D_{k}}$. Since $(y^k)_{k\in\mathbb{N}}$ is quasi-Fej\'er  monotone to $W$, Lemma~\ref{le:pl} implies that $(\alpha_k)_{k\in\mathbb{N}}$ converges. Now, by using  the  first  inequality in \eqref{eq:pnv}, we have  $\|y^{k}-w\|\leq \sqrt{\mu} \alpha_k$, for all $k\in  \mathbb{N}$.  Thus, $(y^k)_{k\in\mathbb{N}}$ is bounded. To prove the last statement, assume that   ${\bar y}, {\hat y} \in  W$   are  cluster points of $(y^k)_{k\in\mathbb{N}}$, and set   $(y^{k_i})_{i\in\mathbb{N}}$ and  $(y^{k_j})_{j\in\mathbb{N}}$  subsequences  of $(y^k)_{k\in\mathbb{N}}$ such that $\lim_{j\to +\infty} y^{k_i} = {\bar y}$ and $\lim_{j\to +\infty} y^{k_j} = {\hat y}$. It follows from the first statement that  $(\|y^{k}- {\bar y}\|_{D_{k}})_{k\in\mathbb{N}}$ and  $(\|y^{k}- {\hat y}\|_{D_{k}})_{k\in\mathbb{N}}$ are convergent. Since $\lim_{k\rightarrow\infty}D_k={\bar D}$, we have  $\lim_{k\rightarrow\infty}\|{\bar y}\|_{D_k}={\|\bar y}\|_{{\bar D}}$ and $\lim_{k\rightarrow\infty}\|{\hat y}\|_{D_k}={\|\hat y}\|_{{\bar D}}$. Hence,  due~to
	$$
		\langle y^k,  D_k({\bar y}- {\hat y})\rangle=\frac{1}{2}(\|y^{k}- {\hat y}\|_{D_{k}}^2-\|y^{k}- {\bar y}\|_{D_{k}} +\| {\bar y}\|_{D_{k}}-\| {\hat y}\|_{D_{k}}),
	$$
	for all $k\in {\mathbb N},$ we conclude that the sequence  $(\langle  y^k,  D_k({\bar y}- {\hat y})\rangle)_{k\in\mathbb{N}}$ converges.  Thus, taking into account that   $\lim_{j\to +\infty} y^{k_i} = {\bar y}$, $\lim_{j\to +\infty} y^{k_j} = {\hat y}$ and  $\lim_{k\rightarrow\infty}D_k={\bar D}$ we obtain that
	$$
		\langle  {\bar y},  {\bar D}({\bar y}- {\hat y})\rangle=\lim_{i\rightarrow\infty} \langle y^{k_i},  D_{k_i}({\bar y}- {\hat y})\rangle=\lim_{j\rightarrow\infty} \langle y^{k_j},  D_{k_j}({\bar y}- {\hat y})\rangle= \langle  {\hat y},  {\bar D}({\bar y}- {\hat y})\rangle.
	$$
	Hence, using \eqref{eq:pnv}, we obtain
	$$
		\frac{1}{\mu}\|{\bar y}- {\hat y}\|^2\leq  \|{\bar y}- {\hat y}\|^2_{\bar D}=  \langle  {\bar y},  {\bar D}({\bar y}- {\hat y})\rangle-  \langle  {\hat y},  {\bar D}({\bar y}- {\hat y})\rangle=0,
	$$
	which implies that ${\bar y}= {\hat y}$. Therefore, due to  $(y^k)_{k\in\mathbb{N}}$ be bounded, we conclude that    $(y^k)_{k\in\mathbb{N}}$ converges to ${\bar y}$.
\end{proof}

\section{Risks measures}

In the world of investments, one of the main points to be observed is the risk of an asset, that is, the possibility of losing money investing in this asset. Off course, there are \textbf{risk free} investments, where there is no possibility of losing money investing in this assets, like Treasury Bonds. The reason that leads an investor to choose a risky asset is the expectation of earning a higher return than investing in a risk free asset that compensates for the possibility of losing the investment.


To build some strategies based in risk we need to define a risk measure. In 1999 Artzner, Delbaen, Eber and Heath (see in \cite{Artzner1999}) defined that,
in order for a function $\mathcal{R}(x)$ to be considered a \emph{coherent risk measure}, it needs to satisfy four
properties:

\begin{enumerate}
\item Subadditivity

\[
\mathcal{R}(x_1+x_2) \leq \mathcal{R}(x_1) + \mathcal{R}(x_2),
\]

that is, joining two portfolios into one is less risky than keeping them
separated.


\item Homogeneity
\[
\mathcal{R}(\lambda x) = \lambda \mathcal{R}(x), \qquad \mbox{se } \lambda
\geq 0.
\]

This property ensures that leverages (borrowing to invest) affect risk proportionately.

\item Monotonicity
\[
\mbox{if } x_1 \prec x_2, \qquad \mbox{then } \mathcal{R}(x_1) \leq \mathcal{R}(x_2).\\
\]

If a asset $x_1$ performs worse than $x_2$ in any
scenario $(x_1 \prec x_2)$, then it means that the asset $x_1$ is
riskier than $x_2$.

\item Translation
\[
\mbox{if } m\in\mathbb{R}, \mbox{then } \mathcal{R}(x+m) = \mathcal{R}(x)-m.
\]
Adding an amount of money, $m$, to a asset decreases your risk by the same amount.
\end{enumerate}


In 2002, Föllmer and Shield (see in \cite{Follmer2002}) proposed a version replacing the
homogeneity and subadditivity, by convexity:
\[
\mathcal{R}(\lambda x_1 + (1-\lambda)x_2) \leq \lambda \mathcal{R}(x_1) + (1-\lambda) \mathcal{R}(x_2).
\]
This condition tells us that diversification does not increase risk.\footnote{Note that a function satisfying conditions 1 and 2 is convex, however a convex function does not always satisfy condition 1. Therefore, there are more functions satisfying Föllmer's condition and Shield, so this is a weaker version.}

Before we look at which functions meet the four conditions above,
we will introduce some important concepts in the study of portfolios.

A portfolio with $n$ assets, $A1, A2, \dots, An$, is a vector of the type
$x^\top = (x_1, x_2, \dots, x_n)$ where each coordinate $x_i$ is the
proportion of the capital invested in the asset $Ai$. We will assume that the capital
is fully allocated, that is, $\sum_{i=1}^n x_i = \textbf{1}^\top x=1$, where
$ \mathbf{1}^\top= (1,1,\dots,1)$. Furthermore, we will consider $x_i\geq 0$ for all $i$, this means that the investor cannot ``\textit{sell without having}'' an asset in the portfolio.\footnote{At first glance, it may seem strange the possibility of an investor selling a good that he does not own, but this practice is common in the market and is called \textbf{short sale}.}


We will denote by $R^\top = (R_1, R_2, \dots R_n)$ the vector of asset returns, where $R_i$ is the return on assets $i$. It follows that the return of the portfolio $x$ is $R(x) = \sum_{i=1}^n x_i R_i$, we can also write in matrix form, $R(x) = x^\top R$.

Consider also $\mu = \mathbb{E}[R]$ the vector of expected returns and $\Sigma = [(R-\mu)(R-\mu)^\top]$ the covariance matrix of asset returns. The expected return of the portfolio is

\[
\begin{aligned}
\mu(x) & = \mathbb{E}[R(x)]=\mathbb{E}[x^\top R] = x^\top \mathbb{E}[R] \\
& = x^\top \mu,
\end{aligned}
\]

and its variance is equal to

\[
\begin{aligned}
\sigma^2 & = \mathbb{E}\Big[\Big(R(x)-\mu(x)\Big)\Big(R(x)-\mu(x)\Big)^\top\ \Big] \\
& = \mathbb{E}\Big[\Big(x^\top R - x^\top \mu \Big)\Big(x^\top R - x^\top \mu \Big)^\top \ \Big] \\
& = \mathbb{E}\Big[x^\top(R-\mu)(R-\mu)^\top x \Big] \\
& = x^\top \mathbb{E}\Big[(R-\mu)(R-\mu)^\top \Big] x \\
& = x^\top \Sigma x.
\end{aligned}
\]
The variance (or standard deviation) of an asset returns is called \textbf{volatility} of the asset.

By definition, the loss of a portfolio is $L(x) = -R(x)$, so we can define different risk measures:

\begin{itemize}
\item Volatility of the loss:
\[
\mathcal{R}(x) = \sigma(L(x)) =\sigma(-R(x)) = \sigma (x).
\]
The standard deviation of loss was defined as a measure of risk to a portfolio by Nobel laureate in economics, Harry
Markowitz in 1952 (see in \cite{Markowitz1952}).

\item VaR (Value at Risk):
\begin{equation}
\mathcal{R}(x) = \mbox{VaR}_\alpha(x) = \inf\{\ell\in\mathbb{R} : P[L(x)\leq \ell]\geq \alpha \},
\end{equation}
where $0<\alpha<1$ and $P[L(x)\leq \ell]$ is the probability of the loss $L(x)$ being less than $\ell$. The Value at Risk, VaR$_\alpha$, is the $100\alpha-$percentile of the loss distribution, that is, the probability of the loss being less than the VaR$_\alpha$ is $\alpha$. The example below shows that in 95\% of cases, losses will be less than VaR$_{95\%}$.

\begin{figure}[!h]
\centering
\includegraphics[width=0.5\textwidth]{figures/VaR}
\caption{VaR$_{95\%}$ as the 95th percentile of the loss distribution.}
\end{figure}

\item CVaR (Conditional Value at Risk):\footnote{CVaR is also known
like \emph{Expected Shortfall} which in a literal translation would be
\emph{Expected Loss}.}

\begin{equation*}
\mathcal{R}(x) =\mbox{ES}_\alpha (x) = \mathbb{E}[L(x)|L(x)\geq \mbox{VaR}_\alpha(x)]
\end{equation*}
The CVaR, $\mbox{ES}_\alpha (x)$, is the expected value of the loss, when it is above the VaR, or
be,
\begin{equation}
\mbox{ES}_\alpha(x) = \frac{1}{1-\alpha}\int_{\mbox{\footnotesize{VaR}}_\alpha(x)}^\infty u p(u)du
\end{equation}
where $p(u)$ is the probability density function of losses. CVaR quantifies the average loss above VaR. The figure below shows the region where the CVaR is calculated.
\begin{figure}[!h]
\centering
\includegraphics[width=0.5\textwidth]{figures/CVaR}
\caption{Loss values used in CVaR calculation.}
\end{figure}
\end{itemize}

While VaR answers the following question:

``\textit{What is the minimum portfolio loss in the $100(1-\alpha)$\% worst case scenarios}?''


CVaR answers the following question:

``\textit{What is the average loss of the portfolio in the $100(1-\alpha)$\% worst case scenarios?}''

\noindent
\parbox{\textwidth}{
\textbf{Notes:}
  \begin{enumerate}
\item The standard deviation does not satisfy condition 4, however, this condition was defined with the perspective of risk to the banking system and is often ignored in the case of portfolio construction.
\item VaR is not a coherent risk measure as it does not satisfy condition 1, which is a problem as portfolio risk can be quite significant.
\item CVaR is a coherent risk measure.
  \end{enumerate}
}

\section{The Gaussian case}

Let us now assume that the returns are normally distributed,
that is, $R\sim N(\mu,\sigma^2)$, where $\mu(x)=x^\top\mu$ and
$\sigma(x)=\sqrt{x^\top \Sigma x}$. We will also consider:
$\phi(z) = \frac{1}{\sqrt{2\pi}}e^{-z^2/2}$ the density function of
probability of the Standard Normal Distribution and $\Phi(x) = \int_{-\infty}^{x}\phi(z) dz$
the standard normal accumulated density function.

By definition we have
$P\Big[L(x)\leq \mbox{VaR}_\alpha(x)\Big]=\alpha$, so
$P\Big[R(x)\leq -\mbox{VaR}_\alpha(x)\Big]=1-\alpha$, which standardizing
we have: \[
P\left[ \frac{R(x)-x^\top\mu}{\sqrt{x^\top \Sigma x}} \leq \frac{-\mbox{VaR}_\alpha(x)- x^\top\mu}{\sqrt{x^\top \Sigma x}}
\right]=1-\alpha.
\]

Follow that \[
\frac{-\mbox{VaR}_\alpha(x)-x^\top\mu}{\sqrt{x^\top \Sigma x}}=\Phi^{-1}(1-\alpha)
\] and since $\Phi^{-1}(\alpha)=-\Phi^{-1}(1-\alpha)$, we have
\begin{equation}\label{eq:var2}
\mbox{VaR}_\alpha(x)=-x^\top\mu+ \Phi^{-1}(\alpha)\sqrt{x^\top \Sigma x}.
\end{equation}

Using the expression obtained for $\mbox{VaR}_\alpha(x)$ it is possible
show that it satisfies conditions 1-4, that is, the Value at Risk is
a coherent measure of risk if asset returns have
Normal Distribution.

The CVaR was defined by:

\[
\mbox{ES}_\alpha(x) = \frac{1}{1-\alpha}\int_{\mbox{VaR}_\alpha(x)}^\infty u p(u)du,
\] considering $p(u)$ the normal density function, we have that:
\[
\mbox{ES}_\alpha(x) = \frac{1}{1-\alpha}\int_{-\mu(x)+ \sigma(x)\Phi^{-1}(\alpha)} ^\infty \frac{u}{\sigma(x)\sqrt{2\pi}} e^{-\frac{1}{2}\Big(\frac{u+\mu(x)}{\sigma (x)}\Big)^2} du,
\]

Making the following change of variable
$t=\frac{u+\sigma(x)}{\sigma(x)}$, we have

\[
\begin{aligned}
\mbox{ES}_\alpha(x) & = \frac{1}{1-\alpha}\int_{\Phi^{-1}(\alpha)}^\infty (-\mu(x)+ \sigma(x)t) \frac{1}{\sqrt{2\pi}} e^{-t^2/2}dt \\
& = -\frac{\mu(x)}{1-\alpha}[\Phi(t)]_{\Phi^{-1}(\alpha)}^\infty +
\frac{\sigma(x)}{(1-\alpha)\sqrt{2\pi}}\int_{\Phi^{-1}(\alpha)}^\infty t e^{-t^2/ 2}dt \\
& =-\mu(x) + \frac{\sigma(x)}{(1-\alpha)\sqrt{2\pi}}\Big[-e^{-t^2/2} \Big] _{\Phi^{-1}(\alpha)}^\infty \\
& = -\mu(x) + \frac{\sigma(x)}{(1-\alpha)\sqrt{2\pi}}e^{-\frac{[\Phi^{-1}(\ alpha)]^2}{2}}
\end{aligned}.
\]

So CVaR can be calculated by \begin{equation}\label{eq:cvar2}
\mbox{ES}_\alpha(x)=-x^\top \mu + \frac{\phi(\Phi^{-1}(\alpha))}{1-\alpha}\sqrt{x^ \top \Sigma x}.
\end{equation}

In the Gaussian world, different risk measures can be calculated
using the expected return and volatility. Note by (\ref{eq:var2}) and
(\ref{eq:cvar2}), that both Var and CVaR are of the form
$-\mu(x)+c\sigma(x)$. In general, we want a portfolio with
positive returns, that is, $\mu(x)\geq0$. So if the manager
portfolio consider a very optimistic expected return on the component
$-\mu(x)$ significantly reduces the risk, which is why the
The market industry uses only the standard deviation as a measure of
risk.

\begin{example}\normalfont Consider three assets $A$, $B$ and $C$ whose values are $\$ 15.00$, $\$ 25.00$ and $\$ 30.00$ respectively. The expected daily return of assets is $30$ bps\footnote{1 bps or one basis point is equivalent to
0.01\% (the hundredth part of 1\%) or 0.0001 in decimal form.}, $50$ bps and $20$ bps and their daily volatilities are $3\%$, $2\%$, and $1\%$ respectively. The asset correlation matrix is given by
\[
\rho = \left(
\begin{array}{rrr}
1.00 & 0.40 & 0.15 \\
0.40 & 1.00 & 0.60 \\
0.15 & 0.60 & 1.00
\end{array}
\right)
\]

Consider now a portfolio formed by $100$ assets $A$, $200$ assets $B$ and $100$ assets $C$. This portfolio is worth $\$ 9,500.00$, being $\$ 1,500.00$ invested from asset $A$, $\$ 5,000.00$ in asset $B$ and $\$ 3,000.00$ in asset $C$, therefore the weights of the assets in this portfolio are $15.79\%$, $52.63\%$, and $31.58\%$ respectively, so $x=(0.1579;\;0.5263;\; 0.3158)$. The expected return on the portfolio is $\mu(x) = 30\times0.1579+50\times0.5263+20\times0.3158$, that is, $\mu(x) = 37$ bps. Using the relationship $\Sigma_{i,j}=\rho_{i,j}\sigma_1\sigma_j$ we find the covariance matrix 
\[
\Sigma = \left(
\begin{array}{llr}
9.0 & 2.4 & 4.5 \\
2.4 & 4.0 & 1.2 \\
4.5 & 1.2 & 1.0
\end{array}
\right)\times10^{-4},
\] since the volatility of the portfolio is given by
$\sigma^2(x) = x^\top \Sigma x$, we have that $\sigma(x) = 1.51\%$.

Considering $\alpha = 0.99%
$, we have $\Phi^{-1}(0.99) = 2.325$ and
$\phi(\Phi^{-1}(0.99))=\phi(2.325)=2.68\%$. Using the equations
$(\ref{eq:var2})$ and $(\ref{eq:cvar2})$ we have

\[
\begin{aligned}
\mathrm{VaR}_{99\%}(x) & = -0.37\% + 2.325 \times 1.51\% = 3.14\% \\
\mathrm{ES}_{99\%}(x) & = -0.37\% + \frac{2.68}{0.01} \times 1.51\% = 3.68\%.
\end{aligned}
\]

Risk can also be expressed in monetary terms, in this case,

\[
\begin{aligned}
\mathrm{VaR}_{99\%}(x) & = 3.14\% \times \$ 9,500.00 = \$ 298.30 \\
\mathrm{ES}_{99\%}(x) & = 3.68\% \times \$ 9,500.00 = \$ 349.60.
\end{aligned}
\]
According to the result of {\rm VaR}, in $99\%$ of the days, the loss will be less than $\$ 298.30$, however in the $1\%$ of the remaining days the {\rm VaR} does not measure how much great can be the loss, this is the role of {\rm CVaR} which indicates that the average loss will be $\$ 349.60$ on the worst $1\%$ days.
\end{example}



