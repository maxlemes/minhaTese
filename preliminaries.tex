\chapter{Preliminaries}  \label{chap:Prel}
%%%%%%%%%%%%%%%%%%%%%%%%%%%%%%%%%%%%%%%%%%%%%%%%%%%%%%%%%%

In this chapter, we introduce  some notation and results used throughout our presentation. We also introduce the concept of risk measure that will be used in Chapter \ref{chap:App}.



First we  consider the  index set  ${\mathbb{N}}:=\{0,1,2,\ldots\}$,  the usual inner  product  $\langle \cdot,\cdot \rangle$ in $\mathbb{R}^n$, and the associated Euclidean norm    $\|\cdot\|$.
Let  $f:\mathbb{R}^n \to \mathbb{R}$ be a differentiable function and $C \subseteq \mathbb{R}^n$. The  gradient $\nabla f$ of $f$ is said to be {\it Lipschitz continuous} in $C$ with constant $L>0$ if $\|\nabla f(x)-\nabla f(y)\|\leq L \|x-y\|$, for all~$x, y\in C$. Combining this definition with the fundamental theorem of calculus, we obtain the following result whose proof can be found in \cite[Proposition A.24]{Bertsekas1999}.

\begin{lemma} \label{Le:derivlipsch}
	Let $f:\mathbb{R}^n \to \mathbb{R}$ be a differentiable function and $C \subseteq \mathbb{R}^n$. Assume that $\nabla f$  is Lipschitz continuous in C with constant $L>0$. Then,
	\[
		f(y) - f(x) - \langle \nabla f(x), y-x \rangle \leq (L/2)\|x-y\|^2,
	\]
	for all~ $x, y\in C$.
\end{lemma}
%Let $f:\mathbb{R}^n \to \mathbb{R}$ be a differentiable function and $C \subseteq \mathbb{R}^n$ be a convex set. The function $f$ is convex on $C$, if $f(y) \geq  f(x) + \langle \nabla f(x), y-x \rangle$, for all $x, y\in C$.
Assume that $C$ is a convex set. The function $f$ is said to be \textit{convex} on $C$, if
\[
	f(y) \geq  f(x) + \langle \nabla f(x), y-x \rangle,
\]
for all $x, y\in C$. We recall  that a point ${\bar{x}} \in C$ is a {\it stationary point} for problem \eqref{eq:OptP} if
\begin{equation} \label{eq:StatPoint}
	\langle \nabla f({\bar{x}}),  x-{\bar{x}}\rangle \geq 0, \qquad \forall ~ x\in  C.
\end{equation}
Consequently, if $f$ is a convex funct{}ion  on $C$, then  \eqref{eq:StatPoint} implies that  $\bar{x} \in \Omega^*$.  We will now present some useful concepts  for the analysis of the sequence generated by the scaled  gradient method, for more details, see \cite{CombettesVu2013}.    For that,  let  $D$ be a $n\times n$ positive definite matrix and $\| \cdot \|_{D} : \mathbb{R}^{n}\rightarrow \mathbb{R}$ be  the norm  defined by
\begin{equation} \label{def:normaD}
	\|d\|_{D}:=\sqrt{\left\langle D d,d\right\rangle},\quad \forall d\in \mathbb{R}^{n}.
\end{equation}
For a fixed  constant $\mu \geq 1$,  {\it denote by  ${\cal D}_{\mu}$  the set of symmetric positive definite matrices $n\times n$ with all eigenvalues contained in the interval $[\frac{1}{\mu}, \mu]$}.  The set ${\cal D}_{\mu}$   is compact. Moreover,  for each $D\in {\cal D}_{\mu}$, it follows that $D^{-1}$ also belongs to $ {\cal D}_{\mu}$. Furthermore,  due to $D\in {\cal D}_{\mu}$,  by \eqref{def:normaD}, we obtain
\begin{equation} \label{eq:pnv}
	\frac{1}{\mu}\|d\|^2\leq \|d\|^2_{D}\leq \mu \|d\|^2, \qquad \forall d\in \mathbb{R}^n.
\end{equation}

Let us recall the the concept of  sequence  quasi-Fej\'er  monotone to a set, introduced in  \cite{CombettesVu2013}.
\begin{definition} \label{def:QuasiFejer}
	Let $(y^k)_{k\in\mathbb{N}}$ be a sequence in $\mathbb{R}^n$ and   $(D_k)_{k\in\mathbb{N}}$ be  a sequence in ${\cal D}_{\mu}$.  The sequence $(y^k)_{k\in\mathbb{N}}$ is said to be quasi-Fej\'er  monotone to a set $W\subset \mathbb{R}^n$ with respect to  $(D_k)_{k\in\mathbb{N}}$ if, there exists  a sequence $(\eta_k)_{k\in\mathbb{N}}\subset [0, +\infty)$ such that $\sum_{k\in \mathbb{N}}\eta_k<\infty$ and for  all $w\in W$, there exists a sequence $(\epsilon_k)_{k\in\mathbb{N}}\subset[0, +\infty)$ such that  $\sum_{k\in \mathbb{N}}\epsilon_k<\infty$, and
	\[
		\|y_{k+1}-w\|_{D_{k+1}}^2\leq (1+\eta_k) \|y^k-w\|_{D_k}^2+\epsilon_k,
	\]
	for    all $k\in \mathbb{N}$.
\end{definition}

The following lemma is useful to study the  quasi-Fej\'er  monotone sequence, its prove can be found in \cite[Lemma 2.2.2]{PolyakLivro1987}.
\begin{lemma} \label{le:pl}
	Let $(\alpha_k)_{k\in\mathbb{N}}$, $(\eta_k)_{k\in\mathbb{N}}$ and $(\epsilon_k)_{k\in\mathbb{N}}$ be a sequences in $[0, +\infty)$  such that $\sum_{k\in \mathbb{N}}\eta_k<\infty$ and $\sum_{k\in \mathbb{N}}\epsilon_k<\infty$. Assume that $\alpha_{k+1}\leq (1 +\eta_k)\alpha_k+\epsilon_k$, for all $k\in {\mathbb N}$. Then, $(\alpha_k)_{k\in\mathbb{N}}$ converges.
\end{lemma}

The main property of  quasi-Fej\'er  monotone sequences is stated in the following. Its proof can be found in \cite[Proposition~3.2 and Theorem 3.3]{CombettesVu2013}. For sake of completeness, we include it here.

\begin{theorem}\label{teo.qf}
	Let $(y^k)_{k\in\mathbb{N}}$ be a sequence in $\mathbb{R}^n$ and   $(D_k)_{k\in\mathbb{N}}$ be  a sequence in ${\cal D}_{\mu}$ such that $\lim_{k\rightarrow\infty}D_k={\bar D}$.   If $(y^k)_{k\in\mathbb{N}}$ is quasi-Fej\'er  monotone to a nonempty set $W\subset  \mathbb{R}^n$ with respect to $(D_k)_{k\in\mathbb{N}}$ then,  for each  $w\in W$,    the sequence  $(\|y^{k}-w\|_{D_{k}})_{k\in\mathbb{N}}$ converges. Furthermore, $(y^k)_{k\in\mathbb{N}}$ is bounded and, if each  cluster point of $(y^k)_{k\in\mathbb{N}}$ belongs to $W$, then there exists ${\bar y}\in W$ such that $\lim_{k\rightarrow\infty}y^k={\bar y}$.
\end{theorem}

\begin{proof}
	Take $w\in W$ and define the sequence $(\alpha_k)_{k\in\mathbb{N}}$, where $\alpha_k:=\|y^{k}-w\|_{D_{k}}$. Since $(y^k)_{k\in\mathbb{N}}$ is quasi-Fej\'er  monotone to $W$, Lemma~\ref{le:pl} implies that $(\alpha_k)_{k\in\mathbb{N}}$ converges. Now, by using  the  first  inequality in \eqref{eq:pnv}, we have  $\|y^{k}-w\|\leq \sqrt{\mu} \alpha_k$, for all $k\in  \mathbb{N}$.  Thus, $(y^k)_{k\in\mathbb{N}}$ is bounded. To prove the last statement, assume that   ${\bar y}, {\hat y} \in  W$   are  cluster points of $(y^k)_{k\in\mathbb{N}}$, and set   $(y^{k_i})_{i\in\mathbb{N}}$ and  $(y^{k_j})_{j\in\mathbb{N}}$  subsequences  of $(y^k)_{k\in\mathbb{N}}$ such that $\lim_{j\to +\infty} y^{k_i} = {\bar y}$ and $\lim_{j\to +\infty} y^{k_j} = {\hat y}$. It follows from the first statement that  $(\|y^{k}- {\bar y}\|_{D_{k}})_{k\in\mathbb{N}}$ and  $(\|y^{k}- {\hat y}\|_{D_{k}})_{k\in\mathbb{N}}$ are convergent. Since $\lim_{k\rightarrow\infty}D_k={\bar D}$, we have  $\lim_{k\rightarrow\infty}\|{\bar y}\|_{D_k}={\|\bar y}\|_{{\bar D}}$ and $\lim_{k\rightarrow\infty}\|{\hat y}\|_{D_k}={\|\hat y}\|_{{\bar D}}$. Hence,  due~to
	$$
		\langle y^k,  D_k({\bar y}- {\hat y})\rangle=\frac{1}{2}(\|y^{k}- {\hat y}\|_{D_{k}}^2-\|y^{k}- {\bar y}\|_{D_{k}} +\| {\bar y}\|_{D_{k}}-\| {\hat y}\|_{D_{k}}),
	$$
	for all $k\in {\mathbb N},$ we conclude that the sequence  $(\langle  y^k,  D_k({\bar y}- {\hat y})\rangle)_{k\in\mathbb{N}}$ converges.  Thus, taking into account that   $\lim_{j\to +\infty} y^{k_i} = {\bar y}$, $\lim_{j\to +\infty} y^{k_j} = {\hat y}$ and  $\lim_{k\rightarrow\infty}D_k={\bar D}$ we obtain that
	$$
		\langle  {\bar y},  {\bar D}({\bar y}- {\hat y})\rangle=\lim_{i\rightarrow\infty} \langle y^{k_i},  D_{k_i}({\bar y}- {\hat y})\rangle=\lim_{j\rightarrow\infty} \langle y^{k_j},  D_{k_j}({\bar y}- {\hat y})\rangle= \langle  {\hat y},  {\bar D}({\bar y}- {\hat y})\rangle.
	$$
	Hence, using \eqref{eq:pnv}, we obtain
	$$
		\frac{1}{\mu}\|{\bar y}- {\hat y}\|^2\leq  \|{\bar y}- {\hat y}\|^2_{\bar D}=  \langle  {\bar y},  {\bar D}({\bar y}- {\hat y})\rangle-  \langle  {\hat y},  {\bar D}({\bar y}- {\hat y})\rangle=0,
	$$
	which implies that ${\bar y}= {\hat y}$. Therefore, due to  $(y^k)_{k\in\mathbb{N}}$ be bounded, we conclude that    $(y^k)_{k\in\mathbb{N}}$ converges to ${\bar y}$.
\end{proof}

